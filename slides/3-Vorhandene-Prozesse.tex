\section{Bereits laufende Prozesse}


\begin{frame}
  \frametitle{Laufende Prozesse}

  \begin{itemize}
    \item es können bereits laufende Prozesse verfolgt werden
    \item Tracing erst ab dem aktuellen Zeitpunkt, nicht „rückwirkend“
    \item Ermittlung der PID
      \begin{itemize}
        \item \texttt{pidof \emph{NAME}}
        \item \texttt{ps aux | grep \emph{NAME}}
      \end{itemize}
    \item Aufruf von \strace{}
      \begin{itemize}
        \item \texttt{strace -p \emph{PID} [-f]}
      \end{itemize}
  \end{itemize}
\end{frame}

\subsection{Konfiguration der Berechtigungen}

\begin{frame}[fragile]
  \frametitle{Konfiguration der Berechtigungen}

  \begin{block}{Wann ist \strace{} laufender Prozesse erlaubt?}
    \begin{itemize}
      \item Root darf alle Prozesse verfolgen
      \item Benutzer darf eigene Prozesse verfolgen
      \item Standardeinstellung vieler Linux-Distributionen verbietet letzteres
    \end{itemize}
  \end{block}

  \pause

  \begin{block}{Temporäres Freischalten}
    \texttt{sysctl kernel.yama.ptrace\_scope=0}
  \end{block}

  \begin{exampleblock}{Dauerhafte Konfiguration: /etc/sysctl.d/10-trace.conf}
    \texttt{kernel.yama.ptrace\_scope=0}
  \end{exampleblock}

\end{frame}

\subsection{Zuordnung von Dateideskriptoren}

\begin{frame}[fragile]
  \frametitle{Zuordnung von Dateideskriptoren}

  \begin{exampleblock}{\texttt{/proc/\emph{PID}/fd}}
    \begin{lstlisting}
lrwx------ 1 bw bw 64  2. Jan 16:35 0 -> /dev/pts/1
lrwx------ 1 bw bw 64  2. Jan 16:35 1 -> /dev/pts/1
lrwx------ 1 bw bw 64  2. Jan 16:35 2 -> /dev/pts/1
lr-x------ 1 bw bw 64  2. Jan 16:35 3 -> /home/bw/devel/strace-talk/sample/somefile
    \end{lstlisting}
  \end{exampleblock}

  \begin{block}{Automatische Zuordnung durch \strace{}}
    \texttt{strace -y}
  \end{block}

\end{frame}