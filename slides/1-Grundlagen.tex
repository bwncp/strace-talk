\section{Grundlagen}

\subsection{User-Mode und Kernel-Mode}

\begin{frame}
  \frametitle{\emph{strace?}}

  \begin{block}{Was ist \emph{strace?}}
    \begin{itemize}
      \item untersucht die Interaktion eines Programms mit dem Betriebssystem
      \item zeigt sog. \emph{Systemaufrufe} mit Argumenten und Rückgabewert an
    \end{itemize}
  \end{block}

  \begin{block}{Vorteile}
    \begin{itemize}
      \item praktisch auf jedem System verfügbar
      \item für ein Tracing-Tool einfach zu verwenden
      \item nicht zu große Performanceinbußen
    \end{itemize}
  \end{block}

\end{frame}

\begin{frame}
  \frametitle{User-Mode und Kernel-Mode}

  \begin{block}{Kernel-Mode}
    \begin{itemize}
      \item Zugriff auf gesamten Speicher und Hardware möglich
      \item auch als \emph{Supervisor-Modus} oder \emph{privilegierter Modus} bezeichnet
      \item wird nur vom Kernel verwendet
    \end{itemize}
  \end{block}

  \begin{block}{User-Mode}
    \begin{itemize}
      \item eingeschränkter Befehlssatz
      \item verwenden normale Programme, auch als Root
    \end{itemize}
  \end{block}
\end{frame}

\subsection{Systemaufrufe}

\begin{frame}
  \frametitle{Systemaufrufe}

  \begin{block}{Was sind Systemaufrufe?}
    Systemaufrufe ermöglichen einem Programm, Dienste des Betriebssystems in Anspruch zu nehmen,
    die den \emph{Kernel-Mode} erfordern.
  \end{block}
  
  \begin{block}{Gruppen von Systemaufrufen}

    \begin{itemize}
      \item Datei- und Verzeichnisverwaltung
      \item Prozessmanagement
      \item Speicherverwaltung
      \item Interprozesskommunikation (IPC)
      \item Netzwerkkommunikation
      \item Behandlung von Signalen
      \item Sonstiges
    \end{itemize}
  \end{block}

\end{frame}

\begin{frame}
  \frametitle{Systemaufrufe für Dateioperationen}

  \begin{itemize}
    \item \texttt{open()} 
      \begin{itemize}
        \item öffnet Datei
        \item gibt \emph{Dateideskriptor} zurück
        \item erwartet Dateinamen und Zugriffsmodus als Parameter
      \end{itemize}

    \item \texttt{read()}
      \begin{itemize}
        \item liest Inhalt von einer Datei
        \item bekommt \emph{Dateideskriptor} (und Speicher) als Argument
        \item gibt Anzahl gelesener Bytes zurück
      \end{itemize}

    \item \texttt{close()}
      \begin{itemize}
        \item gibt Dateideskriptor wieder frei
      \end{itemize}

  \end{itemize}
\end{frame}

\begin{frame}
  \frametitle{Dateideskriptoren}

  \begin{block}{Bedeutung}
    \begin{itemize}
      \item identifiziert geöffnete Datei
      \item Zahl, innerhalb eines Prozesses eindeutig
      \item auch für andere Objekte, insb. \emph{Sockets} (für Netzwerkkommunikation)
    \end{itemize}
  \end{block}

  \begin{block}{Vordefinierte Dateideskriptoren}
    \centering
    \medskip
    \begin{tabular}{|lll|}
      \hline
      \textbf{Dateideskriptor} & \textbf{Symbol} & \textbf{Bedeutung} \\
      \hline
        0      & \texttt{STDIN}    & Standardeingabe (Tastatur) \\
        1      & \texttt{STDOUT}   & Standardausgabe (Terminal) \\
        2      & \texttt{STDERR}   & Standardfehlerausgabe (Terminal) \\
      \hline
    \end{tabular}
    \medskip
  \end{block}
\end{frame}

\begin{frame}
  \frametitle{Parameter und Rückgabewerte}

  \begin{block}{Parameter}
    \begin{itemize}
      \item Zahl und Typ abhängig vom konkreten Systemaufruf
      \item dokumentiert in der jeweiligen Manpage
    \end{itemize}
  \end{block}

  \begin{block}{Rückgabewert}
    \begin{itemize}
      \item immer eine Zahl
      \item $\ge$ 0: Bedeutung abhängig vom Systemaufruf
      \item $<$   0: Fehlercode
      \begin{itemize}
        \item vordefinierte Werte (POSIX, Linux-spezifisch)
        \item dokumentiert in \href{http://man7.org/linux/man-pages/man3/errno.3.html}{\emph{errno(3)}} 
        \item kleine Auswahl auf der nächsten Folie
      \end{itemize}
    \end{itemize}
  \end{block}


\end{frame}

\begin{frame}
  \frametitle{Auswahl verbreiter Fehlercodes}

  \bigskip\centering
  \begin{tabular}{|r@{~~}l|l|}
    \hline
    \multicolumn{2}{|l|}{\textbf{Fehlercode}}        & \textbf{Bedeutung} \\
    \hline
    -1      & \texttt{EPERM}     & Die Operation ist nicht erlaubt \\
    -2      & \texttt{ENOENT}    & Datei oder Verzeichnis nicht gefunden \\
    -5      & \texttt{EIO}       & Eingabe-/Ausgabefehler \\
    -13     & \texttt{EACCES}    & Keine Berechtigung \\
    -20     & \texttt{ENOTDIR}   & Ist kein Verzeichnis \\
    -21     & \texttt{EISDIR}    & Ist ein Verzeichnis \\
    -28     & \texttt{ENOSPC}    & Auf dem Gerät ist kein Speicherplatz mehr verfügbar \\
    \hline
  \end{tabular}


\end{frame}
