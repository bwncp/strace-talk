\section{Bereits laufende Prozesse}

Bisher haben wir \strace{} immer mit einem Programm von Anfang an gestartet. Aber oftmals
ist es auch nützlich, \strace{} auf bereits laufende Prozesse loszulassen, um beispielsweise zu
sehen warum ein Programm hängt.

Zunächst müssen wir dafür die Prozess-ID (PID) des Prozesses herausfinden, entweder mit
\texttt{pidof} oder mit \texttt{ps}. Anschließend kann sich \strace{} mit \texttt{-p \emph{PID}}
an den laufenden Prozess anhängen. Die Option \texttt{-f} (mehrere Threads/Kindprozesse) gilt
analog.

\minisec{Berechtigungen}

Allerdings darf man natürlich nur die Prozesse des eigenen Benutzers verfolgen, nur der
Root-Benutzer darf fremde Tasks tracen. Die Standardkonfiguration vieler Linux-Distributionen
verbietet es allerdings, dem normalen Benutzer fremde Prozesse zu verfolgen. Diese Einstellung
kann mit folgendem Befehl (temporär) geändert werden:

\begin{lstlisting}
  % sudo sysctl kernel.yama.ptrace_scope=0
\end{lstlisting}

Unter \cite{yama} ist diese Einstellung dokumentiert. Über einen Eintrag in
\href{http://man7.org/linux/man-pages/man5/sysctl.conf.5.html}{\emph{sysctl.conf(5)}} kann man
diese Einstellung auch persistent anwenden.


\minisec{Dateideskriptoren zuordnen}

Läuft der Prozess bereits, fehlen einem die entsprechenden \texttt{open()}-Aufrufe, um zu wissen,
welcher Dateideskriptor zu welcher Datei gehört. Diese Information bekommt man leicht aus dem
Proc-Dateisystem: Pro Prozess-ID existiert in \texttt{/proc} ein Verzeichnis, mit jeweils einem
Unterverzeichnis \texttt{fd} (für \emph{file descriptor}). Dort existiert wiederum für jeden
numerischen Dateideskriptor ein symbolischer Link, der zur tatsächlichen Datei zeigt.

Am besten betrachten wir ein Beispiel, nämlich wieder unser leicht modifizertes Programm
\texttt{notfound.py} aus \autoref{fig:sample-listing}. Es wurde lediglich zwischen dem Öffnen
und Schließen der Datei eine kleine Wartezeit eingefügt.

Zunächst finden wir mit \texttt{ps aux|grep notfound.py} die PID des Programms heraus, anschließend
erzeugen wir ein Verzeichnislisting mit \texttt{ls -l /proc/\emph{PID}/fd}:

\begin{lstlisting}
lrwx------ 1 bw bw 64  2. Jan 16:35 0 -> /dev/pts/1
lrwx------ 1 bw bw 64  2. Jan 16:35 1 -> /dev/pts/1
lrwx------ 1 bw bw 64  2. Jan 16:35 2 -> /dev/pts/1
lr-x------ 1 bw bw 64  2. Jan 16:35 3 -> /home/bw/devel/strace-talk/sample/somefile
\end{lstlisting}

0, 1 und zwei sind jeweils mit Standardein-, -ausgabe und -fehlerausgabe vorbelegt (vgl.
\autoref{tab:stdio} während der Dateideskriptor 3 auf die vom Programm selbst geöffnete Datei zeigt.

\tipbox{\sffamily
 Mit dem Schalter \texttt{-y} kann diese Arbeit auch \strace{} für uns übernehmen. Bei jedem
Dateideskriptor wird zusätzlich in spitzen Klammern die zugehörige Datei mit dargestellt!}

