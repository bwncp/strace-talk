\section{Grundlagen}

Bevor wir uns damit beschäftigen, wie man \strace{} verwendet, benötigen wir etwas
Hintergrundwissen, um zu verstehen, was uns das Tool überhaupt anzeigt: Nämlich die sogenannten
\emph{Systemaufrufe} oder engl. \emph{system calls}.

\subsection{User-Mode und Kernel-Mode}

Zunächst müssen wir uns vergegenwärtigen, dass ein "`moderner"'\footnote{Also letztlich alles, was
nach den 1990er Jahren gebaut wurde und etwas größer ist, also kein einfacher Mikrocontroller in
einer Waschmaschine.} Prozessor (mindestens) zwei Modi besitzt: Den \emph{User-Mode} und den
\emph{Kernel-Mode}.

Arbeitet die CPU im \emph{Kernel-Mode,} so ist jeder beliebige Befehl zur Ausführung zugelassen.
Sie kann auf sämtliche Speicherbereiche und damit auch auf sämtliche Hardware zugreifen. Je nach
Architektur nennt man diesen Modus auch \emph{Supervisor-Modus} oder allgemein auch
\emph{privilegierten Modus}. Unter Linux nutzt nur der \emph{Kernel} (Betriebssystemkern) diesen
Modus.

Arbeitet die CPU hingegen im \emph{User-Mode,} so ist nur ein eingeschränkter Befehlssatz zur
Ausführung zugelassen. Es sind also nicht alle Befehle erlaubt, ebenso kann nicht auf alle
Speicherbereiche und auch nicht direkt auf Hardware zugegriffen werden. In diesem Modus werden
normale Programme ausgeführt, auch jene, die mit Administrator-Rechten laufen.

\subsection{Systemaufrufe}

An dieser Stelle kommen nun die Systemaufrufe ins Spiel: Immer wenn ein Programm (im User-Mode)
läuft und Dienste des Betriebssystems benötigt, die nur im Kernel-Mode ausgeführt werden können,
setzt es einen Systemaufruf ab, um diesen Dienst in Anspruch zu nehmen. Natürlich prüft der Kernel
vor der Ausführung, ob der Prozess die entsprechende Berechtigung besitzt.

Es gibt Systemaufrufe für verschiedene Dinge, sie lassen sich in Gruppen
zusammenfassen:

\begin{itemize}
  \item \textbf{Datei- und Verzeichnisverwaltung:} Erstellen, Auflisten, Löschen von Dateien und 
    Verzeichnissen sowie Lesen und Schreiben von Daten. Abfragen und Ändern von Metadaten.
  \item \textbf{Prozessmanagement:} Erzeugen und Terminieren von Prozessen sowie Abfrage deren 
   Status.
  \item \textbf{Speicherverwaltung}
  \item \textbf{Interprozesskommunikation (IPC):} Synchronisation und Kommunikation mit anderen
    Tasks, z.\,B. über Shared Memory, FIFOs oder Semaphoren.
  \item \textbf{Netzwerkkommunikation:} Senden und Empfangen von Daten über Netzwerke mit sog.
   \emph{Sockets,} normalerweise über TCP/IP aber auch über andere Protokolle wie Bluetooth.
  \item \textbf{Behandlung von Signalen:} Signale sind eine Möglichkeit, wie das Betriebssystem
    asynchron mit dem Prozess kommuniziert.
  \item \textbf{Sonstiges:} Darüber hinaus gibt es noch eine Reihe von Systemaufrufen, die sich
    nicht in die Gruppen einfügen, bspw. die Abfrage und das Setzen der Systemzeit.
\end{itemize}

Die Systemaufrufe sind relativ einfach gehalten. Die meisten komplexen Aufgaben setzen sich aus
einfachen Systemaufrufen zusammen, die dann mit Hilfe von Programmierschnittstellen (APIs\footnote{
engl. \emph{Application Programming Interface}}) dem Entwickler mit Bibliotheken zugänglich gemacht
werden. Zum Beispiel gibt es keinen Systemaufruf für das Anzeigen eines Fensters: Stattdessen
kommuniziert die Applikation mit dem \emph{X-Server} über ein festgelegtes Protokoll und verwendet
dafür fertige Programmbibliotheken.

\minisec{Dateioperationen}

An dieser Stelle möchte ich einige Systemaufrufe zum Arbeiten mit Dateien kurz erläutern. Probleme
mit Dateien (z.\,B. falsche Zugriffsrechte) sind ein häufiger Grund, warum Programme nicht
funktionieren und die man schnell mit \strace{} finden kann. Daher beschränke ich mich in diesem
Vortrag auf Dateioperationen, da diese für Nicht-Entwickler am verständlichsten sind.

Um einen Text aus einer Datei zu lesen, kommuniziert ein Programm mit dem Betriebssystem
folgendermaßen:

\begin{enumerate}
  \item \texttt{open()} öffnet die Datei und gibt einen sog. \emph{Dateideskriptor} (engl.
   \emph{file descriptor)} zurück. Hierbei handelt es sich um einen Zahl, die innerhalb eines 
   Prozesses eindeutig ist und die geöffnete Datei referenziert. Unter Unix werden 
   Dateideskriptoren auch für viele andere Objekte verwendet, so zum Beispiel für einen
   \emph{Socket,} einen abstrakten Kanal zur Kommunikation in Netzwerken.

   Neben den Dateideskriptoren als Rückgabewert von Systemaurufen sind die ersten drei 
   Dateideskriptoren bereits standardmäßig geöffnet und haben die in \autoref{tab:stdio} 
   dargestellte Bedeutung.

   Als Parameter erhält der Systemaufruf neben dem \emph{Pfad} der Datei auch den 
   \emph{Zugriffsmodus}, also ob die Datei gelesen oder geschrieben werden soll und ob ggf. die 
   Datei neu erstellt wird wenn sie nicht bereits vorhanden ist.

  \item \texttt{read()} liest den Inhalt von einer Datei. Das erste Argument ist der
   \emph{Dateideskriptor,} danach folgt ein Speicherbereich, in den der Inhalt geschrieben wird.
   Der Rückgabewert ist die Anzahl der gelesenen Bytes.

  \item \texttt{close()} schließt die Datei, gibt den Dateideskriptor also wieder frei.
\end{enumerate}

\begin{table}[htb]
  \centering\small
  \begin{tabular}{|p{1.2cm}ll|}
    \hline
    \textbf{Dateideskriptor} & \textbf{Symbol} & \textbf{Bedeutung} \\
    \hline
      0      & \texttt{STDIN}    & Standardeingabe (Tastatur) \\
      1      & \texttt{STDOUT}   & Standardausgabe (Terminal) \\
      2      & \texttt{STDERR}   & Standardfehlerausgabe (Terminal) \\
    \hline
  \end{tabular}
  \caption{Standardmäßig geöffnete Dateideskriptoren}
  \label{tab:stdio}
\end{table}



Die Dokumentation aller Systemaufrufe findet man in Sektion 2 der Manpages, also bspw.
\href{http://man7.org/linux/man-pages/man2/read.2.html}{\emph{read(2)}}. Zum Anzeigen dient der
Befehl \texttt{man 2 read} oder die Webseite 
\href{http://man7.org/linux/man-pages}{man7.org/linux/man-pages}.

\minisec{Parameter, Rückgabewerte und Fehlercodes}

Die meisten Systemaufrufe haben eine oder mehrere Argumente, deren Typ vom konkreten Aufruf abhängt
und in der jeweiligen Manpage dokumentiert ist.

Wichtig sind neben den Parameter auch die Rückgabewerte der Systemaufrufe. Der Rückgabewert ist
immer eine Zahl. Positive Werte (inkl. der Null) stehen hierbei für das erfolgreiche Ausführen
des Systemaufrufs, deren Bedeutung vom Systemaufruf abhängt. So gibt -- wie eben schon erwähnt --
der Auruf \texttt{read()} die Anzahl der gelesenen Zeichen zurück, 0 steht für das Dateiende. 

Im Fehlerfall wird immer ein negativer Fehlercode zurückgegeben. Es gibt vom Betriebssystem ein 
paar hundert vordefinierte Fehlercodes, die gängigsten sind in
\href{http://man7.org/linux/man-pages/man3/errno.3.html}{\emph{errno(3)}} dokumentiert.
Glücklicherweise kennt \strace{} die Fehlercodes und übersetzt sie in einen symbolischen Bezeichner
und gibt dahinter auch die Bedeutung aus.

\autoref{tab:errno} zeigt einige verbreitete Fehler"=Rückgabewerte mit dessen Bedeutung. Obwohl der
numerische Wert hier aufgelistet wurde, hat er in der Praxis -- anders als etwa unter Windows --
keine Bedeutung. Programme geben, von wenigen Ausnahmen abgesehen, nur den Fehlertext aus.
asf
dasd\afs\d\af\afsd\sfd\fdsa

\begin{table}[b]
  \centering\small
  \begin{tabular}{|r@{~~}l|p{5.2cm}|}
    \hline
    \multicolumn{2}{|l|}{\textbf{Fehlercode}}        & \textbf{Bedeutung} \\
    \hline
    -1      & \texttt{EPERM}     & Die Operation ist nicht erlaubt \\
    -2      & \texttt{ENOENT}    & Datei oder Verzeichnis nicht gefunden \\
    -5      & \texttt{EIO}       & Eingabe-/Ausgabefehler \\
    -13     & \texttt{EACCES}    & Keine Berechtigung \\
    -20     & \texttt{ENOTDIR}   & Ist kein Verzeichnis \\
    -21     & \texttt{EISDIR}    & Ist ein Verzeichnis \\
    -28     & \texttt{ENOSPC}    & Auf dem Gerät ist kein Speicherplatz mehr verfügbar \\
    \hline
  \end{tabular}
  \caption{Auswahl häufiger Fehlercodes mit Bedeutung}
  \label{tab:errno}
\end{table}
