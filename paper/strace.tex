\documentclass[10pt,DIV=14,twocolumn,ngerman,parskip=half]{scrartcl}
\usepackage{babel}
\usepackage[tt=false]{libertine}
\usepackage[svgnames]{xcolor}
\usepackage[unicode=true,
            pdfstartview=FitH,
            pdfusetitle,
            colorlinks=true,
            allcolors=DarkBlue]{hyperref}
\usepackage{microtype}
\hypersetup{pdfauthor={Bernhard Walle, NCP engineering GmbH}}

\urlstyle{sf}

\begin{document}

\title{Fehleranalyse unter Linux mit \emph{strace}}

\author{\href{mailto:bernhard.walle@ncp-e.com}{Bernhard Walle},
        \href{https://www.ncp-e.com}{NCP engineering GmbH}}

\maketitle

\begin{abstract}
    Das Tool \emph{strace} gehört zu den Urgesteinen der Tracing"=Tools unter Linux und anderen
    unixoiden Betriebssystemen. Es ist entweder bereits installiert oder lässt sich über die
    Paketverwaltung leicht nachinstallieren. Es eignet sich 
\end{abstract}



    
\end{document}