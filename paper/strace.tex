\documentclass[10pt,DIV=14,twocolumn,ngerman,parskip=half]{scrartcl}
\usepackage{abstract}
\usepackage{babel}
\usepackage[tt=false]{libertine}
\usepackage[svgnames]{xcolor}
\usepackage[unicode=true,
            pdfstartview=FitH,
            pdfusetitle,
            colorlinks=true,
            allcolors=DarkBlue]{hyperref}
\usepackage{microtype}
\hypersetup{pdfauthor={Bernhard Walle, NCP engineering GmbH}}

\urlstyle{sf}

\begin{document}

\title{Fehleranalyse unter Linux mit \emph{strace}}

\author{\href{mailto:bernhard.walle@ncp-e.com}{Bernhard Walle},
        \href{https://www.ncp-e.com}{NCP engineering GmbH}}

\newcommand{\strace}{\emph{strace}}

\makeatletter
\twocolumn[
  \begin{@twocolumnfalse} 
    \maketitle 
    \begin{abstract} 
      Das Tool \strace{} gehört zu den Urgesteinen der Tracing"=Tools unter Linux und anderen
      unixoiden Betriebssystemen. Es ist entweder bereits installiert oder lässt sich über die
      Paketverwaltung leicht installieren. Mit Hilfe von \strace{} lässt sich die Interaktion
      eines beliebigen Programms mit dem Betriebssystem untersuchen: Es zeigt die sogenannten
      \emph{Systemaufrufe} mit ihren Argumente und Rückgabewerten an.

      Dieser Vortrag ist für Einsteiger gedacht und beginnt daher mit etwas Theorie: Wie
      interagieren Programme mit dem Betriebssystem? Danach wird die Bedienung von \strace{}
      anhand eines einfachen Beispiels erläutert. Zum Schluss folgen verschiedene Nützliche
      Optionen von \strace{} mit ihren Einsatzszenarien.
    \end{abstract}

    \vspace{1cm}
  \end{@twocolumnfalse}
]
\makeatother


\section{Systemaufrufe}

Bevor wir uns damit beschäftigen, wie man \strace{} verwendet, benötigen wir etwas
Hintergrundwissen, um zu verstehen, was uns das Tool überhaupt anzeigt: Nämlich die sogenannten
\emph{Systemaufrufe} oder engl. \emph{system calls}.

\subsection{User-Mode und Kernel-Mode}

Zunächst müssen wir uns vergegenwärtigen, dass ein "`moderner"'\footnote{Also letztlich alles, was
nach den 1990er Jahren gebaut wurde und etwas größer ist, also kein einfacher Mikrocontroller in
einer Waschmaschine.} Prozessor zwei Modi besitzt: Den \emph{User-Mode} und den \emph{Kernel-Mode}.

\begin{description}
  \item[Kernel-Mode]
    Arbeitet die CPU im \emph{Kernel-Mode,} so ist jeder beliebige Befehl zur Ausführung 
    zugelassen. Es kann auf sämtliche Speicherbereiche und damit auch auf sämtliche Hardware 
    zugreifen. Je nach Architektur nennt man diesen Modus auch \emph{Supervisor-Modus} oder
    \emph{privilegierten Modus}.

    Unter Linux nutzt nur der \emph{Kernel} (Betriebssystemkern) diesen Modus.

  \item[User-Mode] Arbeitet die CPU im User-Mode, so ist nur ein eingeschränkter Befehlssatz zur
     Ausführung zugelassen. Es sind also nicht alle Befehle erlaubt, ebenso kann nicht auf alle Speicherbereiche und auch nicht direkt auf Hardware zugegriffen werden. 

     In diesem Modus werden Programme ausgeführt, auch jene, die mit Administrator-Rechten laufen.
\end{description}

\subsection{Systemaufrufe}

An dieser Stelle kommen nun die Systemaufrufe ins Spiel: Immer wenn ein Programm (im User-Mode)
läuft und Dienste des Betriebssystems benötigt, setzt es einen Systemaufruf ab, um diesen Dienst
in Anspruch zu nehmen.

Es gibt Systemaufrufe für verschiedene Dinge, sie lassen sich in verschiedenen Gruppen
zusammenfassen:

\begin{itemize}
  \item \textbf{Datei u. Verzeichnisverwaltung:} Erstellen, Auflisten, Löschen von Dateien und 
    Verzeichnissen sowie Lesen und Schreiben von Daten. Abfragen und Ändern von Metadaten.
  \item \textbf{Prozessverwaltung:} Erzeugen und Terminieren von Prozessen sowie Abfrage deren 
   Status.
  \item \textbf{Interprozesskommunikation (IPC):} Synchronisation und Kommunikation mit anderen Tasks, z.\,B. über Shared Memory, FIFOs, Mutexe.
  \item \textbf{Netzwerkkommunikation:} Senden und Empfangen von Daten über Netzwerke mit sog.
   \emph{Sockets,} normalerweise über TCP/IP aber auch über andere Protokolle wie Bluetooth.
  \item \textbf{Sonstiges:} Darüber hinaus gibt es noch eine Reihe von Systemaufrufen, die sich
    nicht in die Gruppen einfügen, bspw. die Abfrage und das Setzen der Systemzeit.
\end{itemize}

Die Systemaufrufe sind relativ einfach gehalten. Die meisten komplexen Aufgaben setzen sich aus
einfachen Systemaufrufen zusammen, die dann mit Hilfe von Programmierschnittstellen (APIs\footnote{
engl. \emph{Application Programming Interface}}) dem Entwickler mit Bibliotheken zugänglich gemacht
werden.

\minisec{Dateioperationen}

An dieser Stelle möchte ich einige Systemaufrufe zum Arbeiten mit Dateien kurz erläutern. Probleme
mit Dateien (z.\,B. falsche Zugriffsrechte) sind ein häufiger Grund, warum Programme nicht
funktionieren und die man schnell mit \strace{} finden kann. Daher beschränke ich mich in diesem
Vortrag auf Dateioperationen, da diese für Nicht-Entwickler am verständlichsten sind.

Um einen Text aus einer Datei zu lesen, kommuniziert ein Programm mit dem Betriebssystem
folgendermaßen:

\begin{enumerate}
  \item \texttt{open()} öffnet die Datei und gibt einen sog. \emph{Dateideskriptor} zurück. Hierbei
   handelt es sich um eine Zahl, die innerhalb eines Prozesses eindeutig ist und die geöffnete
   Datei referenziert. Unter Unix werden Dateideskriptoren auch für viele andere Objekte verwendet, 
   so zum Beispiel für einen \emph{Socket,} einen abstrakten Kanal zur Kommunikation in Netzwerken.

   Als Parameter erhält der Systemaufruf neben dem \emph{Pfad} der Datei auch den 
   \emph{Zugriffsmodus}, also ob die Datei gelesen oder geschrieben werden soll und ob ggf. die 
   Datei neu erstellt wird wenn sie nicht bereits vorhanden ist.

  \item \texttt{read()} liest den Inhalt von einer Datei. Das erste Argument ist der
   \emph{Dateideskriptor,} danach folgt ein Speicherbereich, in den der Inhalt geschrieben wird.
   Der Rückgabewert ist die Anzahl der gelesenen Bytes.

  \item \texttt{close()} schließt die Datei, gibt den Dateideskriptor also wieder frei.
\end{enumerate}

Die Dokumentation aller Systemaufrufe findet man in Sektion der Manpages, also bspw.
\emph{read(2)}. Zum Anzeigen dient der Befehl \texttt{man 2 read}.

    
\end{document}